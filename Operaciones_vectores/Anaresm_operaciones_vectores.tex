\documentclass[11pt,a4paper]{article}
\usepackage[utf8]{inputenc}
\usepackage{amsmath}
\usepackage{amsfonts}
\usepackage{amssymb}
\usepackage[document]{ragged2e}
\usepackage{apacite}
\usepackage{polynom}
\usepackage{geometry}
\usepackage{graphicx}
\graphicspath{ {C:/Users/ASTRONAUTA/Downloads/latex} }
 \geometry{
 a4paper,
 total={170mm,257mm},
 left=20mm,
 top=20mm,
 }
 
\begin{document}
\begin{center}
{\Huge Universidad Autónoma de Guerrero}
\vspace{2cm}

\begin{huge}
UAGRO\\
\vspace{2cm}
Actividad 2. Operaciones con vectores.
\end{huge}
\end{center}
\vspace{3cm}
\begin{flushleft}
\begin{LARGE}
Alumno: Andrés Nares Monroy  \\
Asignatura: Geometría Analítica \\
Carrera: Ingeniería en Computación \\
\end{LARGE}
\end{flushleft}
\newpage



\begin{justify}


\noindent
1)Actividad 3 - Verifica las propiedades S2 y S3\\

$$(\overline{a}+\overline{b})+\overline{c} = \overline{a} + (\overline{b}+\overline{c})$$

Sean los vectores $\overline{a} = (a_1, a_2), \overline{b} = (b_1, b_2), \overline{c} = (c_1, c_2)$, entonces se tiene la igualdad $((a_1, a_2)+(b_1, b_2))+(c_1, c_2) = (a_1, a_2) + ((b_1, b_2)+(c_1, c_2)) $, entonces $(a_1+b_1, a_2+b_2)+(c_1, c_2) = (a_1, a_2) + ((b_1+c_1, b_2+c_2))$, entonces $(a_1+b_1+c_1, a_2+b_2+c_2)= (a_1+b_1+c_1, a_2+b_2+c_2)$. Por lo tanto se cumple la propiedad asociativa en vectores.\\

$$ \overline{a}+\overline{b} = \overline{b}+\overline{a} $$

Sean los vectores $\overline{a} = (a_1, a_2), \overline{b} = (b_1, b_2) $, entonces se tiene la igualdad $(a_1,a_2)+(b_1,b_2)=(b_1, b_2)+(a_1, a_2)$, entonces $(a_1+b_1,a_2+b_2)=(b_1+a_1, b_2+a_2)$. Por propiedad conmutativa de números reales se tiene $a+b = b+a $ para todo $ a,b \in \mathbb{R}$. Por lo tanto se cumple la propiedad conmutativa en vectores.\\

\noindent
2) Actividad 4 - Que representan sus diagonales para el polígono\\

\noindent
La intersección de las diagonales representa la suma de los vectores. El segmento que une el vértice de inicio con el vértice de las diagonales representa la longitud de de ambos vectores sumados.\\

\noindent
2) Utilizando conocimientos anteriores demuestre lo sig.\\

$$ \overline{a}+\overline{b} = \overline{c} \leftrightarrow \overline{a} = \overline{c}-\overline{b} $$

Sea la representación gráfica anterior se tiene la igualdad $\overline{a}+\overline{b} = \overline{c}$, entonces por existencia y unicidad de un vector se tiene $-\overline{c} +\overline{c}= -\overline{a}-\overline{b}= \overline{a}+\overline{b}$, entonces $-\overline{a} = -\overline{c}+\overline{b}$, entonces $\overline{a}=  \overline{c}-\overline{b}$.\\

Sea $\overline{a} = \overline{c}-\overline{b}$, entonces por igualdad en una ecuación se tiene $\overline{a}+\overline{b} = \overline{c}-\overline{b}+\overline{b}$. Por lo tanto $\overline{a}+\overline{b} = \overline{c} \leftrightarrow \overline{a} = \overline{c}-\overline{b}$.\\

\noindent
3) Los vectores a y b satisfacen: $|\overline{a}|=4, |\overline{b}| = 3, (a\angle b) = \dfrac{\pi}{2}$.\\

\noindent
a+b \\

\includegraphics[width=15cm]{vect2}\\

\noindent
a-b\\

\includegraphics[width=15cm]{sumVect}\\

\noindent
-a+b\\

\noindent
\includegraphics[width=15cm]{sumVect2}\\

\noindent
-a-b\\

\noindent
\includegraphics[width=15cm]{sumVect3}\\


\noindent
Actividad V\\

\noindent
$\overline{a}$\\

\noindent
\includegraphics[width=15cm]{mutate}\\

\noindent
$2\overline{a}$\\
\includegraphics[width=15cm]{mutate2a}\\

\noindent
$1/2\overline{a}$\\
\includegraphics[width=15cm]{mutate3}\\

\noindent
-3/2$\overline{a}$\\
\includegraphics[width=15cm]{mutate4}\\

\noindent
$\sqrt{2}\overline{a}$\\
\includegraphics[width=15cm]{mutate5}\\


\noindent
Actividad VI \\


\noindent
1) \\

\noindent
\includegraphics[width=15cm]{6.1}\\
\includegraphics[width=15cm]{6.11}\\


\noindent
2)\\

$$ (-1)\overline{a} = -\overline{a} $$

\noindent
Sea la propiedad P5, entonces $1(\overline{a}=\overline{a}$, entonces $-1(\overline{a}=-\overline{a}$.Por lo tanto $-1(\overline{a}=-\overline{a}$.\\

$$ (-\alpha)\overline{a} = -(\alpha \overline{a}) $$

\noindent
Sea la propiedad P4. Se sustituye $\beta = -1$, entonces $\alpha(-1\overline{a}) = (\alpha -1)\overline{a}$, entonces $-1(\alpha\overline{a}) = (\alpha -1)\overline{a}$. Por lo tanto  $(-\alpha)\overline{a} = -(\alpha \overline{a})$.\\

$$ \alpha \overline{a} = \overline{0} \leftrightarrow \overline{a} = \overline{0} \vee \alpha = 0 $$

\noindent 
Sea  $ \alpha\overline{a} = \overline{0} $, entonces $ \alpha\overline{a} + \alpha \overline{b} = \overline{0} + \alpha \overline{b}$, entonces por existencia y unicidad de elemento neutro se tiene $\alpha\overline{a} + \alpha \overline{b} =\alpha \overline{b}$, entonces por la propiedad P2 se tiene $\alpha\overline{a} + \alpha \overline{b} = \alpha (\overline{a}+\overline{b}) = \alpha \overline{b}$, entonces por la propiedad p4 $(\alpha-\alpha) (\overline{a}+\overline{b}) = (\alpha-\alpha) \overline{b}$, entonces $(\overline{a}+\overline{b}) =\overline{b}$, entonces para cualquier $\overline{b},\overline{a} \in V$ se tiene la igualdad anterior y por la propiedad de unicidad y existencia de elemento neutro se tiene $\overline{a} = \overline{0}$.\\

\noindent
Sea  $ \alpha\overline{a} = \overline{0} $, entonces  $ \alpha\overline{a}+ \beta \overline{a} = \overline{0} + \beta \overline{a}$, entonces por existencia y unicidad de elemento neutro y la propiedad P3 se tiene $\alpha\overline{a} + \beta \overline{a} =(\alpha + \beta) \overline{a}= \beta \overline{a}$, pero por la propiedad de unicidad y existencia de elemento neutro se tiene dado su igualdad que tienen un mismo único elemento neutro por lo que $\alpha = 0$.\\

\noindent
Sea $\alpha = 0 \vee \overline{a}=\overline{0}$ entonces el modulo $|\alpha||\overline{0}| = ||\alpha|\overline{0}| $, entonces $\alpha \overline{a}$ tiene modulo cero. Por lo tanto 

\noindent
Demuestra que si $\overline{a} \in V$ y $\overline{a} \neq \overline{0}$ entonces $\overline{a}_0$ se puede expresar:\\

$$ \overline{a}_0 = \dfrac{1}{'\left|\overline{a}\right|}\overline{a}$$

Sea el vector $\overline{a}_0$ y el modulo de $|\overline{a}|$, entonces $||a|\overline{a}_0| =|a||\overline{a}_0| = |a|$. Sean ambos vectores con el mismo sentido y dirección, entonces $|a|\overline{a}_0= a$, entonces $\overline{a}_0 = \dfrac{a}{|a|}$, entonces por la existencia de elemento neutro y asociativa respecto a los valores escalares se tiene $\overline{a}_0 = \dfrac{1}{|a|}a$. Por lo tanto  $ \overline{a}_0 = \dfrac{1}{'\left|\overline{a}\right|}\overline{a}$.\\


\noindent
4)\\

$$\alpha \neq 0, \alpha \overline{a} = \overline{b} \leftrightarrow \overline{a} = \dfrac{1}{\alpha} \overline{b}  $$

\noindent
Sea $\alpha \neq 0, \alpha \overline{a} = \overline{b}$, entonces por la propiedad asociativa respecto a números reales se tiene $\alpha (\dfrac{1}{\alpha} \overline{a}) =\dfrac{1}{\alpha} \overline{b}$, entonces $(\alpha \dfrac{1}{\alpha}) \overline{a} = \dfrac{1}{\alpha} \overline{b}$, entonces $ \overline{a} = \dfrac{1}{\alpha} \overline{b}$.\\

\noindent
Sea $\overline{a} = \dfrac{1}{\alpha} \overline{b}$, entonces por la propiedad asociativa respecto a números reales se tiene $\alpha \overline{a} = \alpha (\dfrac{1}{\alpha} \overline{b})$, entonces $\alpha \overline{a} = (\alpha \dfrac{1}{\alpha}) \overline{b}$, entonces $\alpha \overline{a} =  \overline{b}$. Por lo tanto $\alpha \neq 0, \alpha \overline{a} = \overline{b} \leftrightarrow \overline{a} = \dfrac{1}{\alpha} \overline{b}  $.\\

\noindent
5)\\

$$ \overline{a} = |\overline{a}|\overline{a}_0 $$

Sea $\overline{a}_0 = \dfrac{1}{'\left|\overline{a}\right|}\overline{a}$, entonces por asociatividad en números reales se tiene $|\overline{a}|\overline{a}_0 = |\overline{a}|\dfrac{1}{'\left|\overline{a}\right|}\overline{a}$, entonces $|\overline{a}|\overline{a}_0 = (|\overline{a}|\dfrac{1}{'\left|\overline{a}\right|})\overline{a}$. Por lo tanto $ \overline{a} = |\overline{a}|\overline{a}_0 $.\\

\noindent
Actividades VII\\

\noindent
1) Dado $\overline{a} = \alpha \overline{b}$. Prueba que $\overline{a} // \overline{b}$\\

\noindent
Sea por definición que un vector $\overline{b}$ tiene la misma dirección que un vector $\alpha \overline{b}$ para cualquiera $\alpha ,\overline{b}$, entonces $\alpha, \beta$ tienen la misma dirección.Por lo tanto son paralelos.\\

\noindent
2) Demuestra el reciproco.\\

\noindent
Sea $\overline{a} // \overline{b}$ entonces tienen la misma dirección, entonces si tienen diferente sentido se utiliza un escalar negativo para cambiar el signo de alguno de los vectores.\\

\noindent
Sean ambos vectores de misma dirección y sentido,entonces por definición de vector asociado se tiene $a_0=\dfrac{1}{|a|}\overline{a} = \dfrac{1}{|b|}\overline{b}$, entonces $\dfrac{1}{|b|}\overline{b} = \dfrac{1}{|a|}\overline{a}$, entonces  $\dfrac{|a|}{|b|}\overline{b} = \overline{a}$. Por lo tanto si $a // b \rightarrow \overline{a} = \alpha \overline{b}$.\\

\noindent
3) Halla números $\alpha, \beta $ tales que:\\

$$ \overline{a} = \alpha \overline{b} $$

\noindent
Sea $\alpha = -1/2$, entonces $\overline{a} = \alpha \overline{b}$, entonces $\overline{a} = -1/2 \overline{b}$.\\

$$ \overline{b} = \beta \overline{a} $$

Sea $\beta = -2$, entonces $\overline{b} = \beta \overline{a}$, entonces $ \overline{b} = 2 \overline{a} $\\

\noindent
4) Si $\overline{OP}, \overline{O_1P_1}$ son vectores iguales, situados sobre rectas paralelas no coincidente, demuestra que los vectores $\overline{OO_1}$ y $\overline{PP_1}$ son iguales.\\

\noindent
Sea el vector $\overline{P_1 O}$, entonces la suma de los vectores $\overline{O_1P_1}+\overline{P_1 O} = \overline{OO_1}$. Realizando la suma de los vectores $\overline{P_1 O}+\overline{OP} = \overline{PP_1}$. Se obtiene que ambas sumas de vectores son idénticas en relación a que $\overline{O_1P_1} = \overline{OP}$,entonces $\overline{PP_1} = \overline{OO_1}$.Por lo tanto tienen las mismas direcciones y son paralelas.\\

\noindent
5) Si $\overline{a}, \overline{b}$ son vectores representados en la figura.\\


\noindent
a) \includegraphics[width=15cm]{6.1}\\  
 

\noindent
Actividades VIII\\

\noindent
Dados dos vectores paralelos. ¿Qué ángulos pueden formar?\\

\noindent
Ninguno. Puesto que tienen el mismo ángulo.\\

\noindent
a+b\\
\includegraphics[width=15cm]{vector_img}\\

\noindent
a-b\\
\includegraphics[width=15cm]{vect_img2}\\

\noindent
b-a\\
\includegraphics[width=15cm]{vect-img3}\\

\noindent
$2a-1/3 \overline{b}$\\
\includegraphics[width=15cm]{geo_img}\\

\noindent
$\overline{b}-\sqrt{2}\overline{a}$\\
\includegraphics[width=15cm]{geo_img2}\\

\noindent
$2\overline{a}+\overline{b}$\\
\includegraphics[width=15cm]{geo_img3}\\

\noindent
$-1/2 \overline{a}+\overline{b}$\\
\includegraphics[width=15cm]{geo_img4}\\

\noindent
$-3\overline{a}-\dfrac{2}{3}\overline{b}$\\
\includegraphics[width=15cm]{geo_img4}\\

\noindent
$-3\overline{a}-\dfrac{2}{3}\overline{b}$\\
\includegraphics[width=15cm]{geo_img5}\\

\noindent
$\overline{a}+\dfrac{3}{2}\overline{b}$\\
\includegraphics[width=15cm]{geo_img6}\\

\noindent
$\overline{c}$ tal que $2\overline{a}-\overline{b}+\overline{c} = \overline{0}$\\
\includegraphics[width=15cm]{geo_img7}\\

\noindent
\noindent
Suponiendo $\overline{a}, \overline{b}$ no nulos y paralelos, entonces al considerar ambos vectores de la forma $\overline{a} = |\overline{a}|\overline{a}_0$ y $\overline{b} = |\overline{b}|\overline{b}_0$, entonces se tiene la igualdad $|\overline{b}||\overline{a}|\overline{a}_0 = |\overline{a}||\overline{b}|\overline{b}_0$, entonces $\dfrac{|\overline{a}|\overline{a}_0}{|\overline{a}|} = \dfrac{|\overline{b}|\overline{b}_0}{|\overline{b}|}$, entonces $\dfrac{\overline{a}}{|\overline{a}|} =\dfrac{\overline{b}}{|\overline{b}|}$. Por lo tanto mientras se cumpla la hipotesis se cumple la consecuencia.\\

\noindent
La segunda ecuación se cumple cuando cuando tienen sentido contrario\\

\noindent
Actividad IX\\

\noindent
$$ \overline{a}\overline{b} = \overline{b}\overline{a} $$

Sea por definición $|\overline{a}||\overline{b}| \cos \alpha = |\overline{b}||\overline{a}| \cos \alpha$. Por conmutatividad de números reales la multiplicación de escalares es igual. Por lo tanto $ \overline{a}\overline{b} = \overline{b}\overline{a} $.\\

$$ \lambda (\overline{a}\overline{b}) = (\lambda \overline{a})\overline{b} = \overline{a}(\lambda \overline{b}) $$


\noindent
Sea $\lambda (\overline{a}\overline{b}) = \lambda |\overline{a}||\overline{b}|\cos \alpha$.\\

\noindent
$(\lambda \overline{a})\overline{b} = |\lambda \overline{a}||\overline{b}| \cos \alpha = |\lambda || \overline{a}||\overline{b}| \cos \alpha$.\\


\end{justify}



\newpage

\nocite{tippens2007fisica}

\bibliographystyle{apacite}
\bibliography{references}



\end{document}